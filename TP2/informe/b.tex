\newpage{}
\section{Problema B: Buenos gráficos}
\textbf{Peso del ejercicio: 9}
\subsection{Descripción del problema}
	Dados los precios de $A$ acciones durante $D$ dias, se nos 
	pregunta la mínima cantidad de gráficos para graficar los 
	precios de estas acciones en el tiempo de tal manera que 
	los precios no se intersecten. Si los precios de dos acciones
	no se intersectan pueden dibujarse en un mismo gráfico.
	
	
\subsection{Solución al problema}
	Nos construimos un grafo donde los vertices representan 
	las acciones y el vertice $i$ esta conectada con el vertice $j$ si
	los precios de la acci\'on $j$ son estrictamente mayores que los precios de la 
	acci\'on $i$. Simbolicamente $G=(V,E)$ donde $V=\{a_1,\dots ,a_A\}$ y si
	$P_{i,t}$ denota al precio de la accio\'n i-esima a tiempo $t$ entonces
	$(i,j) \in E$ si y solo si $P_{i,t}<P_{j,t}$ para todo $1\leq t\leq D$.
	
	
	
	El grafo es dirigido claramente, es transitivo pues si $(i,j)$,$(j,k)$ $\in$ $E$
	entonces $P_{i,t}<P_{j,t}$ y $P_{j,t}<P_{k,t}$ para todo $1\leq t\leq D$ entonces
	$P_{i,t}<P_{k,t}$ para todo $1\leq t\leq D$ entonces $(i,k)$ $\in$ $E$. También
	es aciclico pues si $i$ y $j$ estan en un ciclo por transitividad
	vale que $(i,j)$,$(j,i)$ $\in$ $E$ entonces  $P_{i,t}<P_{j,t}$ y $P_{j,t}<P_{i,t}$, absurdo.
	Luego G es un DAG.
	
	
	Si llamamos $r$ a la cantidad mínima de caminos para cubrir el DAG
	y $s$ a la cantidad mínima de gráficos para repartir las acciones
	sin que haya solapamiento, quiero ver que $r=s$.
	
	%por que el menor cubrimiento por cadenas es lo que quiero??
	Notamos $g_1,\dots ,g_s$ a los gráficos, a cada gráfico $g_k$ le corresponde
	un conjunto de acciones $a_{j_1},\dots ,a_{j_{N_k}}$ que no se solapan
	por lo que puedo suponer que estan ordenadas de menor a mayor, con respecto
	a los precios, entonces
	$c_k=((j_1,j_2),\dots ,(j_{N_k-1},j_{N_k}))$ es un camino en G. De esta manera me construyo
	$c_1,\dots ,c_s$ caminos y cubren G pues toda acción esta algún 
	gráfico entonces toda acción esta en algun camino.
	Luego $r\leq s$.
	
	Sea $c_1,\dots ,c_r$ el menor cubrimiento por caminos del
	grafo. Si $a_{j_1},\dots ,a_{j_{N_k}}$ son los elementos de
	camino $c_k$ ordenados de menor a mayor entonces $P_{i,t}<P_{i+1,t}$
	para todo $1\leq t\leq D$ y para todo $1\leq i\leq N_k-1$, 
	luego me construyo el grafico $g_k$ con estas acciones, no se
	solapan pues la diferencia entre los precios es positiva.
	Luego $s\leq r$.
	Luego $r=s$.
	
\subsection{Algoritmo}
	%~ Por lo visto en la clase de flujo para encontrar la cantidad mínima 
	%~ de caminos para cubrir un DAG de $n$ vertices
	%~ $v_1,\dots ,v_n$, me construyo un grafo bipartito duplicando los vertices
	%~ $u_1,\dots ,u_n$ y conecto $v_i$ con $u_j$ si $v_i$ estaba conectado con 
	%~ $v_j$ en el DAG y calculo matching maximo de flujo resultante $m$ sobre este grafo. Luego
	%~ la minima cantidad de caminos para cubrir el DAG es $n-m$.
	
		
	%~ Por lo visto en la clase de flujo para encontrar la cantidad mínima 
	%~ de caminos para cubrir un DAG de $n$ vertices
	%~ $v_1,\dots ,v_n$ lo represento como un grafo bipartito duplicando
	%~ los vertices originales. Ahora tengo dos conjuntos de vertices
	%~ $\{ v_1,\dots ,v_n\} $ y $\{ u_1,\dots ,u_n\} $. Conecto $v_i$ 
	%~ con $u_j$ si $v_i$ esta conectado con $v_j$ en el DAG. Si $m$ es el
	%~ matching maximo de este grafo bipartito, que encontramos usando
	%~ el algoritmo de Edmonds Karp como vimos en clase, entonces
	%~ la minima cantidad de caminos para cubrir el DAG es $n-m$.
	
	
	Dadas $n$ acciones $ v_1,\dots ,v_n$ me construyo el grafo bipartito
	duplicando los vertices originales. Ahora tengo dos conjuntos
	de vertices $\{ v_1,\dots ,v_n\} $ y $\{ u_1,\dots ,u_n\} $ y
	$(i,j)$ es una arista del grafo si $P_{i,t}<P_{j,t}$ para todo $1\leq t\leq D$.
	Si $m$ es el matching maximo de este grafo bipartito, que encontramos
	usando el algoritmo de Edmonds Karp, entonces por los visto en la
	clase de flujo la minima cantidad de caminos para cubrir $G$, el DAG 
	de las acciones, es $n-m$.
	
	
	Detectar cuales vertices se conectan cuesta $\mathcal{O}(A^2D)$ y como el grafo
	bipartito tiene $2A$ nodos y $A^2$ aristas la complejidad de 
	calcular matching maximo es $\mathcal{O}(A^3)$.
	En total el algoritmo tiene una complejidad de $\mathcal{O}(A^2(A+D))$ como
	se pedia en el enunciado.
\newpage
\subsection{Código de la solución}
\lstset{inputencoding=utf8/latin1}
\lstinputlisting[numbers = left]{../src/ej2/ej2.cpp}
