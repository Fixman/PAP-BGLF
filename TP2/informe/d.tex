\newpage{}
\section{Problema D: Desocupando del pabellón}
\textbf{Peso del ejercicio: X}
\subsection{Descripción del problema}

El problema puede modelarse de la siguiente forma si consideramos como vértices a las aulas y como aristas dirigidas a los pasillos. Dado un grafo dirigido de $\textbf{A}$ vértices y $\textbf{P}$ aristas, responder $\textbf{Q}$ preguntas de la forma: "¿Existe un camino de $v_1$ a $v_2$ y otro de $v_2$ a $v_1$?"
\subsection{Soluciones al problema}

En lo que sigue llamaremos $\mathcal{A}$ (aulas) al conjunto de vértices y $\mathcal{P}$ (pasillos) al conjunto de aristas. Por enunciado tenemos que $\# \mathcal{A} = \textbf{A}$ y que $\# \mathcal{P} = \textbf{P}$

\subsubsection{Enfoque naïve}

Una vez obtenida la representación del grafo con listas de adyacencia, podríamos responder a cada query lanzando un algoritmo de recorrido de grafo ("BFS" o "DFS" por ejemplo) desde $v_1$ chequeando si alcanza a $v_2$, y otro desde $v_2$ chequeando si se alcanza a $v_1$. 

De esa forma tendríamos una complejidad de  $\mathcal{O} \left (\textbf{A} + \textbf{P} \right ) $ para leer la entrada y luego  $\mathcal{O} \left (\textbf{A} + \textbf{P} \right )$ para responder cada query. Obteniendo en total un algoritmo de complejidad $\mathcal{O} \left (\textbf{A} + \textbf{P} + \textbf{Q} \cdot \left ( \textbf{A} + \textbf{P} \right ) \right )$, que claramente no satisface la complejidad pedida por el enunciado.

\subsubsection{Enfoque con componentes fuertemente conexas}

Notemos primero que nada que no se realiza ningún cambio en el grafo entre las queries, eso nos da la pauta de que si realizamos un precómputo en $\mathcal{O}(\textbf{A} + \textbf{P})$ que nos permita resolver cada query en $\mathcal{O}(1)$, entonces tendríamos un algoritmo con complejidad $\mathcal{O}(\textbf{A} + \textbf{P} + \textbf{Q})$ como pide el enunciado.

Estudiemos lo que se nos pide en las queries, para entender entonces qué precomputar (aunque el título de la sección ya es un spoiler). Dados dos vértices $v_1$ y $v_2$, queremos saber si existe un camino de $v_1$ a $v_2$ y viceversa, pero por lo visto en clase esto ocurre si y solo si $v_1$ y $v_2$ están en una misma componente fuertemente conexa del grafo.

Por ende, si calculamos en $\mathcal{O}(\textbf{A} + \textbf{P})$ a qué componente fuertemente conexa pertence cada nodo, y nos guardamos ese resultado en un arreglo llamado $\texttt{scc}$ de tamaño $\textbf{A}$, responder una query se reduce a responder si ocurre que $\texttt{scc}[v_2] = \texttt{scc}[v_2]$, lo cual se realiza en $\mathcal{O}(1)$ y cumple con lo estudiado en el primer párrafo.

\subsection{Algoritmo y Análisis de Complejidad}

\begin{enumerate}
	\item Leemos $\textbf{A}$ y $\textbf{P}$. Complejidad $\mathcal{O}(1)$
	\item Leemos las $\textbf{P}$ aristas y las guardamos como lista de adyacencia del grafo original, y también del grafo transpuesto. Creamos \texttt{orden} como una cola de dos puntas (solo para evidenciar que podemos apilar al frente de la cola) vacía. Complejidad $\mathcal{O}(2\cdot \textbf{P}) + \mathcal{O}(1) = \mathcal{O}(\textbf{P})$
	\item Corremos DFS, y al finalizar la visita a un nodo, empujamos dicho nodo al frente de $\texttt{orden}$. Como este último paso agrega una operación $\mathcal{O}(1)$ por cada vértice, en total tenemos la misma complejidad de DFS. $\mathcal{O}(\textbf{A} + \textbf{P})$ 
	\item Corremos DFS, pero en el orden dado por $\texttt{orden}$ y en el grafo transpuesto. Además en el ciclo central del DFS vamos actualizando un contador que aumentamos cada vez que lanzamos una búsqueda nueva, y cada vez que terminamos la visita de un nodo, guardamos en $\texttt{scc}$ a qué componente pertence el nodo. Nuevamente la complejidad de esto es $\mathcal{O}(\textbf{A} + \textbf{P})$, pue solo agregamos un contador y la actualización de un lugar en el arreglo, que se hacen a lo sumo una vez por nodo.
	\item Leemos $\textbf{Q}$. Complejidad $\mathcal{O}(1)$
	\item Leemos las $\textbf{Q}$ queries y en cada query, leemos a $v_1$ y $v_2$. Si $\texttt{scc}[v_1] = \texttt{scc}[v_2]$ imprimimos "S", si no, imprimimos "N". Complejidad $\mathcal{O}(\textbf{Q})$, pues se responde en $\mathcal{O}(1)$ cada query.
\end{enumerate}

\begin{itemize}
	\item \textbf{Observación} : Notar que los puntos 3 y 4 corresponden al Algoritmo de Kosaraju visto en clase.
\end{itemize}

Nos queda una complejidad de $\mathcal{O}(\underset{1 \text{ y } 2}{\underbrace{\textbf{P}}} + \underset{3}{\underbrace{(\textbf{A} + \textbf{P})}} + \underset{4}{\underbrace{(\textbf{A} + \textbf{P})}} + \underset{5 \text{ y } 6}{\underbrace{\textbf{Q})}}$. Lo cual nos da como complejidad final $\mathcal{O}(\textbf{A} + \textbf{P} + \textbf{Q})$, que cumple con la complejidad pedida en el trabajo práctico.	 
\newpage
\subsection{Código de la solución}
\lstinputlisting[numbers = left]{../src/ej4/ej4.cpp}
