\newpage{}
\section{Problema C: Cortes programados}
\textbf{Peso del ejercicio: 9}
\subsection{Descripción del problema}
Dada una ciudad con $N$ esquinas y $M$ calles bidereccionales que conectan 
pares de esquinas, donde se puede viajar entre cualquier par de esquinas 
usando las calles, se pide responder una serie de queries de varios tipos: 

\begin{itemize}
\item Tipo A: dadas 2 esquinas $e_1$ y $e_2$, dar la cantidad de calles tales que, 
si cortáramos únicamente esa calle, impediría viajar desde $e_1$ hasa $e_2$. 
\item Tipo B: dada una calle, devolver si existen al menos 2 esquinas entre las 
que dejaría de haber camino si cortáramos la calle. 
\item Tipo C: dada una esquina $e$, devolver la cantidad de esquinas $e_2$ tales que, 
de cortar una sola calle cualquiera, seguiría habiendo camino entre $e_1$ y $e_2$.
\end{itemize}

Se nos pide diseñar un algoritmo que resuelva el problema en complejidad temporal 
$\bigO(M + MQ_A + Q_B + Q_C)$, donde $Q_A$, $Q_B$ y $Q_C$ son las 
queries de tipo A, B y C respectivamente. 

\subsection{Soluciones al problema}
Modelaremos la ciudad como un grafo, de la manera clásica, donde los nodos 
serán las esquinas de la ciudad, y las aristas serán las calles (esto 
tiene sentido porque las calles conectan \textit{pares} de esquinas). 
Como sabemos que entre todo par de esquinas hay un camino, esto quiere 
decir que el grafo que obtenemos es conexo, y por tanto 
además $\bigO(N) \in \bigO(M)$.

Para resolver el problema, lo que haremos será calcular los \textbf{puentes} 
en el grafo resultante, y responderemos las queries utilizando información 
que obtendremos en base a los puentes. 

Veamos cómo se pueden responder los distintos tipos de queries usando 
los puentes el grafo: \\

\textbf{Tipo A:}

La query nos habla de aristas que al ser removidas individualmente 
impedirían llegar desde $e_1$ hasta $e_2$. De alguna manera nos habla de los puentes 
del grafo, porque esas son las aristas que al ser removidas aumentan la cantidad de 
componentes conexas y cortan caminos entre nodos. 

De modo que la query nos habla de los 
puentes, pero no de todos, sino de \textbf{los puentes que se encuentran 
en un camino simple desde $e_1$ hasta $e_2$}. 

Si bien puede haber más de un camino simple entre $e_1$ y $e_2$, \textbf{no pueden 
existir dos que usen distintos puentes}. Si existiesen dos, entonces 
tendría que haber un ciclo conteniendo a esos puentes, lo cual es un absurdo. 
De modo que los puentes en un camino simple de $e_1$ a $e_2$ son únicos, y es 
claro que cumplen con la condición que buscamos porque necesariamente al remover 
uno quedan separados $e_1$ y $e_2$, de lo contrario habría otro camino que los 
mantiene conectados y esto haría que el puente pertenezca a un ciclo, que también 
es un absurdo. 

Para responder las queries entonces teniendo los puentes del grafo precomputados 
podemos encontrar un camino simple de $e_1$ a $e_2$ con un BFS, y luego 
iterar sobre dicho camino y contar la cantidad de puentes en el. El BFS 
siempre encuentra camino (porque el grafo es conexo) y tiene complejidad 
$\bigO(N+M) = \bigO(M)$. Luego $Q_A$ queries de tipo A tienen complejidad 
$\bigO(MQ_A)$. \\

\textbf{Tipo B:}

La query nos pide devolver para una arista dada, si al 
removerla hay al menos 2 nodos que quedan desconectados. Como todos 
los nodos empiezan conectados por ser el grafo conexo, la única manera de que 
al remover una arista se desconecten nodos es que aumente la cantidad de componentes 
conexas, y \textbf{esto ocurre únicamente al remover un puente} (por definición). 

Teniendo los puentes del grafo precomputados, dada una arista podemos responder 
en $\bigO(1)$ si es o no un puente, luego $Q_B$ queries de tipo B tienen complejidad 
$\bigO(Q_B)$. \\

\textbf{Tipo C:}

En esta query queremos dado un nodo $e_1$, responder cuántos otros nodos 
se mantienen conectados a $e_1$ sin importar qué arista sea removida (sólamente una arista). 

Es claro que los nodos alcanzables desde $e_1$ con un camino simple que contenga 
puentes no pueden estar incluidos en la respuesta, dado que si cortamos alguno de los puentes en 
dicho camino el nodo queda desconectado (por lo que vimos con las queries de Tipo A, el 
conjunto de puentes en un camino simple entre 2 nodos es único). 

Por otro lado, todos los nodos alcanzables desde $e_1$ con un camino simple que no 
utilicen ningún puente estarán incluidos en la respuesta. 
Al ser todas las aristas no puentes, están cada una contenida en 
algún ciclo simple, removiendo cualquiera de ellas, podemos reconstruir otro camino que 
los mantenga conectados usando el ciclo. 

Esto último es equivalente a \textbf{considerar las componentes conexas del grafo resultante 
de quitar todos los puentes}. Los nodos en cada componente son alcanzables entre sí 
cumpliendo la condición de la query, y dos nodos de distintas componentes no cumplen 
la condición por lo visto antes. 

Tenemos entonces una equivalencia entre los nodos que queremos contar y su alcanzabilidad 
en el grafo. Para responder las queries, podemos tener precomputadas las componentes conexas 
del grafo sin puentes. Sabiendo para cada nodo a qué componente de estas pertenece y el tamaño 
de las componentes, podemos responder cada query en $\bigO(1)$. La respuesta será el tamaño 
de dicha componente restado 1, porque debemos descontar al propio nodo por el cual estamos preguntando. 
Luego $Q_C$ queries de tipo B tienen complejidad $\bigO(Q_C)$. \\

\subsection{Algoritmo}
Describimos a continuación el algoritmo completo en forma de pseudocódigo 
que resuelve el problema en base a lo presentado en el análisis. 

\begin{algorithm}[H]
dfsPuentes(0,0,0)\;
\For{cada nodo no visitado por dfsComponentes}{
componente $\leftarrow$ \{\}\;
dfsComponentes(nodo, componente)\;
	\For{cada nodo de componente}{
		tamaño[nodo] $\leftarrow$ componente\;
	}
}
\For{cada query}{
\If{tipo A}{
c $\leftarrow$ bfsCaminoSimple(u,v)\;
\textbf{return} cantidad de puentes en c\;
}
\If{tipo B}{
\textbf{return} si la arista es puente o no\;
}
\If{tipo C}{
\textbf{return} tamaño[nodo]-1\;
}
}
\caption{Algoritmo general}
\end{algorithm}

El siguiente es el pseudocódigo de \texttt{dfsPuentes}, algoritmo que se encarga de calcular 
los puentes del grafo. 

\begin{algorithm}[H]
\KwData{v, d, padre}
profundidad[v] = low[v] = d\;
\For{cada vecino w de v distinto del padre}{
    \eIf{profundidad[w] == -1}{
        dfsPuentes(w,d+1,v)\;
        low[v] = min(low[v], low[w])\;
        \If{low[w] >= profundidad[w]}{
            marcar (v,w) como puente\;
        }
    }{
        low[v] = min(low[v], low[w])\;
    }
}
\caption{\texttt{dfsPuentes}}
\end{algorithm}

Por último mostramos el pseudocódigo de \texttt{dfsComponentes}, para calcular 
las componentes conexas del grafo sin puentes. El algoritmo es una variación 
de un DFS, con la única diferencia de evitamos visitar un nodo si llegamos a el 
por un arista puente. 

\begin{algorithm}[H]
\KwData{v, componente}
visitar v\;
agregar v a componente\;
\For{cada vecino w de v no visitado}{
	\If{la arista (v,w) no es puente}{
		dfsComponentes(w, componente)\;
	}
}
\caption{\texttt{dfsComponentes}}
\end{algorithm}

\newpage
\subsection{Código de la solución}
\lstinputlisting[numbers = left]{../src/ej3/ej3.cpp}
