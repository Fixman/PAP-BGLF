\newpage{}
\section{Problema C: Cortes programados}
\textbf{Peso del ejercicio: 9}
\subsection{Descripción del problema}
Dada una ciudad con $N$ esquinas y $M$ calles bidereccionales que conectan 
pares de esquinas, donde se puede viajar entre cualquier par de esquinas 
usando las calles, se pide responder una serie de queries de varios tipos: 

\begin{itemize}
\item Tipo A: dadas 2 esquinas $e_1$ y $e_2$, dar la cantidad de calles tales que, 
si cortáramos únicamente esa calle, impediría viajar desde $e_1$ hasa $e_2$. 
\item Tipo B: dada una calle, devolver si existen al menos 2 esquinas entre las 
que dejaría de haber camino si cortáramos la calle. 
\item Tipo C: dada una esquina $e$, devolver la cantidad de esquinas $e_2$ tales que, 
de cortar una sola calle cualquiera, seguiría habiendo camino entre $e_1$ y $e_2$.
\end{itemize}

Se nos pide diseñar un algoritmo que resuelva el problema en complejidad temporal 
$\bigO(M + MQ_A + Q_B + Q_C)$, donde $Q_A$, $Q_B$ y $Q_C$ son las 
queries de tipo A, B y C respectivamente. 

\subsection{Soluciones al problema}
Modelaremos la ciudad como un grafo, de la manera clásica, donde los nodos 
serán las esquinas de la ciudad, y las aristas serán las calles (esto 
tiene sentido porque las calles conectan \textit{pares} de esquinas). 
Como sabemos que entre todo par de esquinas hay un camino, esto quiere 
decir que el grafo que obtenemos es conexo, y por tanto 
además $\bigO(N) \in \bigO(M)$.

Para resolver el problema, lo que haremos será calcular los \textbf{puentes} 
en el grafo resultante, y responderemos las queries utilizando información 
que obtendremos en base a los puentes. 

Veamos cómo se pueden responder los distintos tipos de queries usando 
los puentes el grafo: 
\begin{itemize}
\item Tipo A:
\item Tipo B:
\item Tipo C:
\end{itemize}

\subsection{Algoritmo}
\newpage
\subsection{Código de la solución}
\lstinputlisting[numbers = left]{../src/c/c.cpp}
