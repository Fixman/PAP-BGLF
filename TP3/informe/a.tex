\newpage{}
\section{Problema A: Ayudando a los gorilas}
\textbf{Peso del ejercicio: 10}
\subsection{Descripción del problema}
	El problema consta de dado un string T de largo $n$ y un string P de 
	largo $m\leq n$, decidir si P es substring de T.
	
\subsection{Solución al problema}
	Ya que se nos pide que la soluci\'on tenga una complejidad temporal
	lineal en la longitud del string T,ie. $O(n)$, vamos a implementar el algoritmo
	de Knuth, Morris y Pratt, KMP. Pero antes veamos el algoritmo naive
	ya que alli aparece naturalmente la problematica que soluciona KMP.
	
\subsection{Algoritmo Naive}
	El algoritmo naive para ver si P es substring de T es recorrer
	T lugar a lugar y en cada posici\'on comparar si P matchea o no. M\'as
	formalmente verificamos si existe $0\leq s \leq n-m$ tal que 
	$P[i]=T[s+i]$ para todo $1\leq i \leq m$.
	
	\incmargin{1em}
	\restylealgo{boxed}\linesnumbered
	\begin{algorithm}
		\SetKwInOut{Input}{input}
		\SetKwInOut{Output}{output}
		\caption{matchingNaive}
		\Input{String T y un string P}
		\Output{Valor booleano}
		\BlankLine
	
		$n\leftarrow$longitud T\;
		$m\leftarrow$longitud P\\
		\For{$i\leftarrow 1$ hasta $n-m+1$}{
			\For{$j\leftarrow1$ hasta $m$}{
				\If{T[i+j] es distinto de P[j]}{
					Salir de bucle interno\;
				}
				\If{Llege al ultimo caracter de P}{
					return True\;
				}
			}
		}
		return False\;
	\end{algorithm}
	\decmargin{1em}
	
	
	Si $m=n/2$ entonces el algoritmo hace $\frac{n^2}{4}$ iteraciones
	en el bucle, por lo cual tiene una compejidad $O(n^2)$. Observemos
	que la funcion que nos da el siguiente caracter desde el cual
	comparar nuestro patron es simplemente trasladar un caracter hacia
	la derecha lo cual desaprovecha cualquier comparaci\'on que hayamos
	hecho. El algoritmo de KMP arregla esto preprocesando 
	la estructura del patron para dar una funci\'on de salto mas eficiente.


\subsection{Algoritmo de Knuth, Morris y Pratt}
	El algoritmo de KMP aprovecha las comparaciones ya hechas del patron
	P contra el texto T habiendo prepocesado una tabla de bordes del 
	patron. 
	
	
	Concretamente definimos $P_q=P[1,\dots ,q]$ el arreglo
	de los primeros $q$ caracteres del patr\'on P y 
	$borde[q]=max\{k:$ $k<q$ y $P_k$ es subfijo de $P_q\}$ luego $borde[q]$ es la 
	longitud del prefijo de P mas grande que es sufijo propio de $P_q$.
	
	
	Ejemplo de tabla de bordes para P='abcabcda'
	
	
	\begin{table}[h]
		\centering
		\begin{tabular}[h]{|c|c |c |c |c |c |c |c |c |}
			\hline
			i & 1 & 2 & 3 & 4 & 5 & 6 & 7 & 8\\ \hline
			P & a &b&c&a&b&c&d&a \\ \hline
			borde & 0& 0& 0& 1& 2 &3& 0& 1 \\ \hline
		\end{tabular}
	\end{table}
	
	
	Para aprovechar las comparaciones ya calculadas, por ejemplo si ya
	se que para algun $s$ tengo que $P_q=T[s+1,\dots ,s+q]$ pero $P[q+1]\neq T[s+q+1]$ 
	entonces busco el siguiente indice $s'$ desde el cual voy a empezar a comparar. 
	Este indice $s'$ cumple que es el menor tal que $s'>s$ y $P_k=T[s'+1,\dots ,s'+k]$ con
	con $s'+k=s+q$, es decir $s'$ es el primer indice potencialmente valido despues de $s$ y
	de nuestra definici\'on de $borde[q]$ se sigue que $s'=s+(q-borde[q])$.
	
	Entonces el algoritmo de KMP consta de dos partes, por un lado el 
	calculo de la tabla de bordes y por el otro el algorimto de matching
	que recorre el string sobre todos los potenciales $s$.
	
	\subsubsection{Calculo de la tabla de bordes}
	El siguiente presudocodigo calcula la tabla de bordes de un patron P.
	
	\incmargin{1em}
	\restylealgo{boxed}\linesnumbered
	\begin{algorithm}
	\SetKwInOut{Input}{input}
	\SetKwInOut{Output}{output}
	\caption{calculoBordes}
	\Input{Arrglo de caracteres $P$}
	\Output{Arreglo de enteros}
	\BlankLine

	$m\leftarrow$ longitud de P\;
	$borde[1] \leftarrow 0$\;
	$k\leftarrow 0$\;
	\For{$q\leftarrow 2$ hasta $m$}{
		\While{k>0 y P[k+1] es distinto de P[q]}{
			$k\leftarrow borde[k]$\;
		}
		\If{P[k+1] es igual a P[q]}{
			$k\leftarrow k+1$\;
		}
		$borde[q] \leftarrow k$\;
	}
	return borde\;
	\end{algorithm}
	\decmargin{1em}

	\paragraph{Analicemos la correctitud del calculo de la tabla de bordes,} 
		para ello defino $borde^*[q] = \{borde[q],borde^2[q],\dots , borde^t[q]\}$ 
		donde $borde^k[q] = borde[borde^{k-1}[q]]$ y $t$ es el primer
		$k$ tal que $borde^k[q]=0$. Ademas para $q=2,\dots ,m$ defino
		$E_{q-1}=\{k:$ $k<q-1$ y $P_{k+1}$ es sufijo de $P_q  \}$. Via el
		lema 32.5 de \cite{cormen} tenemos que $E_{q-1}=\{k\in borde^*[q-1]:P[k+1]=P[q]\}$.
		
		
		En la linea 2 seteamos $borde[1]=0$ lo cual es correcto por 
		definicion pues $borde[1]<1$ y $P_0$ es el string vacio que
		es sufijo de $P_1$. En cada iteracion del bucle For tenemos que
		$k=borde[q-1]$ por las lineas 2 y 3 y 11. Las lineas $5-9$ ajustan
		$k$ para sea el valor de $borde[q]$ correspondiente a la iteracion
		y las lineas $5-6$ buscan entre los valores en $borde^*[q-1]$ hasta
		que encuentra alguno para el cual valga que $P[k+1]=P[1]$, en ese
		momento $k$ es el valor mas grande de $E_{q-1}$ entonces por el
		Coroloario 32.7 de \cite{cormen} podemos setear $borde[q]=k+1$. Si tal
		$k$ no se encuentra entonces $k=0$ por la linea $6$. Si $P[1]=P[q]$ entonces
		se setea $k=1$ y $borde[q]=1$ de otra manera $k=0$ y $borde[q]=0$. Luego
		borde se setea correctamente..
		
	\paragraph{Veamos que la complejidad temporal del calculo de borde}
		resulta lineal en $m$, para ello basta ver
		que las lineas $5-9$ se ejecutan en total $O(m)$ veces, pero esto
		es inmediato pues $k$ empieza valiendo 0 entonces en cada iteraci\'on
		del For aumenta en uno en la linea 9 y descrese en la linea 6 pues $k>borde[k]$,
		pero siempre k es no negativo.
		Entonces en el peor caso por cada vez que $k$ aumenta se mete en el while
		y como $k$ aumenta como mucho m veces entonces se mete en el While
		como mucho m veces entonces se ejecuta en total $O(m)$ veces.
	
	
	\subsubsection{Matching}
	\incmargin{1em}
	\restylealgo{boxed}\linesnumbered
	\begin{algorithm}[h]
	\SetKwInOut{Input}{input}
	\SetKwInOut{Output}{output}
	\caption{knuthMorrisPratt}
	\Input{Dos arreglos de enteros P y T}
	\Output{Entero}
	\BlankLine
			$n\leftarrow$longitud T\;
			$m\leftarrow$longitud P\;
			$borde\leftarrow$calculoBordes(P)\;
			$q\leftarrow 0$\;
			\For{$i\leftarrow 1$ hasta $n$}{
				\While{q>0 y P[q+1] distinto de T[i]}{
					$q\leftarrow borde[q]$\;
				}
				\If{P[q+1]=T[i]}{
					$q\leftarrow q+1$\;
				}
				\If{q=m}{
					return $i$\;
				}
			}
			return $-1$ \;
	\end{algorithm}
	\decmargin{1em}
	
	\paragraph{Verfiquemos la correctitud,} veamos que P es substring de T si y solo 
		si $KMP(T,P)\neq -1$.
		
		
		Si $P$ es un substring entonces existe m\'inimo $s$ tal que $P=T[s+1,\dots ,s+m]$.
		
		Notamos $q_i$ el valor de $q$ al principio de la $i-esima$ iteraci\'on del For.
		
		Luego si $i=s+1$ y $q_i=0$ en las iteraciones siguiente del bucle For se
		van a ejecutar las lineas 9-10 hasta que $q_{i+m}=m-1$.
		 
		Si en cambio $i=s+1$ y $q_i>0$ entonces como $s$ es el m\'inimo
		indice desde el cual coincide se sigue que existe $1\leq j<m$	
		tal que $P[q_i+j]\neq T[s+j]$ luego en esta iteraci\'on $q_{i+j-1}$ decrece
		porque entra en el While. Esto se repite 
		hasta que en alguna iteraci\'on tengo que $i=s+j$ y $q_i=j-1$, 
		no puede suceder que $q_i<j-1$ pues luego de $m-j$ iteraciones
		se sigue que $q_{s+m}<m-1$ lo cual es absurdo por hipotesis.
	
		
		Reciprocamente si $j=KMP(T,P)\neq -1$ entonces se tiene que
		$P[m]=T[j]$ y $q_j=m-1$. Si $m=1$ ya estoy, sino sigo para atras y afirmo 
		que $P[m-1]=T[j-1]$ pues sino analicemos el valor de $q_{j-1}$. Por
		un lado tengo que $q_{j-1}<m$ pues sino habria salido de la ejecucion
		en la iteracion $j-2$. Si $q_{j-1}=m-1$ tengo que $P[m-1]\neq T[j]$ porque
		sino salgo en este paso entonces $q_{j-1}$ entra en el while y como
		$P[m-1]\neq T[j-1]$ tengo que $q_{j-1}<m-2$. Absurdo pues como mucho
		por iteracion aumenta en 1 y $q_j=m-1$.
	
	\paragraph{Analisis de complejidad}
		Para ver que la complejidad de KMP es $O(n)$  basta ver que
		las lineas 6-11 se ejecutan $O(n)$ veces. 
		Esto es claro pues el valor de $q$ solo se modifica en las lineas
		7 y 10, mas aun por cada iteracion del While debe haberse
		ejecutado la linea 10. Por lo cual en el peor de los casos
		si contamos que por cada vez que se ejecuto la linea 10 se ejecuto
		la linea 7 y como la linea 10 se puede ejecutar como mucho $n$ veces
		entonces la linea 7 se puede ejecutar como mucho $n$ veces, entonces
		en total el algoritmo tarda $O(n)$.
	
	\subsubsection{Conclusi\'on}
		La complejidad total de la implementacion resulta $O(n)+O(m)=O(n)$
		,pues $m\leq n$, como queriamos.
\newpage
\subsection{Código de la solución}
\lstset{inputencoding=utf8/latin1}
\lstinputlisting[numbers = left]{../src/ej1/ej1.cpp}

