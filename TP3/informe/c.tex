\newpage{}
\section{Problema C: Ciencia argentina}
\textbf{Peso del ejercicio: 8}
\subsection{Descripción del problema}
Dada una tabla $M$ de dimensiones $C \times A$, que contiene montos por 
cargo y por antigüedad docente, se sabe que el sueldo de un docente 
con cargo $c$ y antigüedad $a$ cobra todos los montos $M_{i,j}$ 
con $1 \leq i \leq c$ y $1 \leq j \leq a$. 

Se nos dan $Q$ queries, en las que en cada una debemos responder lo siguiente: 
dados un cargo $c_1$ y antigüedad $a_1$ de un docente, y el cargo $c_2$ y antigüedad $a_2$ a 
la que aspira llegar a fin de año, ?`qué parte de su sueldo cobraría si tuviera cargo y antigüedad 
$c_2$ y $a_2$, que no podría cobrar teniendo cargo $c_1$ (aún si obtuviese antigüedad $a_2$) o teniendo 
antigüedad $a_2$ (aún si obtuviese el cargo $c_2$)?

Es decir, cada respuesta es la suma de los montos que cobra un docente con cargo y antigüedad 
$c_2$ y $a_2$, que no pueden cobrar docentes que tengan cargo $c_1$ o antigüedad $a_1$.

El algoritmo debe tener complejidad $\mathcal{O}(AC + Q)$. 

\subsection{Soluciones al problema}

En base a lo que nos describe el enunciado, asumiremos primero por un momento 
que $c_1 \leq c_2$ y $a_1 \leq a_2$, dado que un docente buscaría escalar en 
cargo y antigüedad en lugar de bajar. 

Siendo así, como cada query nos pide 
la suma de los montos que un docente con cargo $c_2$ y antigüedad $a_2$ puede cobrar, 
pero no uno con cargo $c_1$ (con antigüedad desde 1 hasta $a_2$) o 
antigüedad $a_2$ (con cargo desde 1 hasta $c_2$), en definitiva lo que se nos pide 
es la suma de los montos en el rectángulo que va desde $(c_1+1,a_1+1)$ hasta $(c_2, a_2)$ 
(agregamos 1 a los índices porque no queremos incluir montos de $c_1$ o $a_1$ en ese rango). 

Dicho de otra forma, lo que queremos para cada query es: 

\begin{equation}
Q(c_1, a_1, c_2, a_2) = \sum\limits_{i = c_1+1}^{c_2} \sum\limits_{j = a_1+1}^{a_2} M_{i,j}
\ref{res-query}
\end{equation}

\subsubsection*{Solución naïve}
Si para cada query computamos la suma requerida simplemente iterando sobre los elementos 
de $M$ que necesitamos, cada query la estaríamos respondiend en $\mathcal{O}(AC)$, porque potencialmente 
una query podría ser de una rectángulo que ocupe toda la tabla. Necesitaríamos una solución 
que nos permita computar cada query en $\mathcal{O}(1)$. 

Tampoco podemos precomputar todas las queries posibles porque en total son $\mathcal{O}((AC)^2)$ 
posibles combinaciones, así que tampoco es viable.  

\subsubsection*{Solución con tablita aditiva}
Lo que podemos hacer es utilizar la estrucura de ``tablita aditiva'' sobre $M$ 
para poder responder las queries rápidamente. 

La idea sería tener precómputado en otra tabla $T$ de las mismas dimensiones, en cada 
posición $(i,j)$, la suma de todos los elementos de $M$ en el rectángulo que va desde el $(1,1)$ 
hasta $(i,j)$, es decir 

\begin{equation}
T_{i,j} = \sum\limits_{a = 1}^{i} \sum\limits_{b = 1}^{j} M_{a,b}
\end{equation}

Teniendo esto, podremos responder cada query en $\mathcal{O}(1)$, veremos ahora 
cómo, y cómo precomputar esta tabla. 


\subsection{Algoritmo}

Veamos primero cómo se calculas los valores de $T$ que usaremos para responder las queries. 

\subsubsection*{Precómputo de la tabla aditiva}
Para llenar la tabla $T$ usaremos un algoritmo basado en programación dinámica, 
donde usaremos valores de $T$ de menor índice para calcular los de índices más grandes. 

Queremos en $T_{i,j}$ la suma antes mencionada, podemos separar en algunos casos: 
\begin{itemize}
\item Si $i = j = 1$, el rectángulo que estamos queriendo sumar tiene solamente el 
elemento $(1,1)$, entonces simplemente $T_{1,1} = M_{1,1}$. 

\item Si $i = 1$ y $j > 1$, es decir, estamos parados sobre la primer fila de $T$. 
Si tenemos ya calculado $T_{1,j-1}$, esto es la suma de la primer fila hasta la 
columna $j-1$. Si a esto le agregamos el elemento $M_{1,j}$, tenemos exactamente la 
suma de la primer fila hasta la columna $j$, que es lo que queremos en $T_{1,j}$. 

Análogamente, podemos resolver el caso $i > 1$ y $j = 1$, es decir la primer columna, 
de una manera similar. 

\item El caso general, donde $i > 1$ y $j > 1$. Nuevamente podemos usar valores de $T$ más 
chicos que ya hayamos calculado. Primero que nada necesitaremos el elemento $M_{i,j}$. 
A esto podemos agregarle la suma de los elementos que están a su izquierda y arriba, 
en principio podemos obtenerlos de $T_{i,j-1}$ y $T_{i-1,j}$. Sumando estas tres tenemos: 

\begin{equation}
M_{i,j} + T_{i,j-1} + T_{i-1,j} = M_{i,j} + \sum\limits_{a = 1}^{i} \sum\limits_{b = 1}^{j-1} M_{a,b} + \sum\limits_{a = 1}^{i-1} \sum\limits_{b = 1}^{j} M_{a,b}
\end{equation}


\begin{equation}
M_{i,j} + T_{i,j-1} + T_{i-1,j} = \sum\limits_{a = 1}^{i} \sum\limits_{b = 1}^{j} M_{a,b} + \sum\limits_{a = 1}^{i-1} \sum\limits_{b = 1}^{j-1} M_{a,b}
\end{equation}

Con estas 3 cosas obtenemos el rango de valores que queremos, pero además nos está sobrando 
todo el rectángulo que va desde el $(1,1)$ hasta el $(i-1,j-1)$. Pero esto es exactamente lo 
que tenemos en $T_{i-1,j-1}$, de modo que podemos restarlo y obtenemos 

\begin{equation}
M_{i,j} + T_{i,j-1} + T_{i-1,j} - T_{i-1,j-1} = \sum\limits_{a = 1}^{i} \sum\limits_{b = 1}^{j} M_{a,b}
\end{equation}

\end{itemize}

Resumiendo entonces tenemos que la función para calcular los valores de $T$ es 

\begin{align*}
T_{i,j} =
	&\begin{cases}
		M_{i,j} & \text{si $i=1$ y $j=1$} \\
		M_{i,j} + T_{i,j-1} & \text{si $i=1$ y $j > 1$} \\
		M_{i,j} + T_{i-1,j} & \text{si $j=1$ y $i > 1$} \\
		M_{i,j} + T_{i,j-1} + T_{i-1,j} - T_{i-,j-1} & \text{si $j>1$ y $i > 1$} \\
	\end{cases} \\
\end{align*}

Se puede ver que cada valor de $T$ depende únicamente de valores de $T$ 
con índices menores, o valores de $M$, de modo que podemos calcularlos en 
orden y cada uno tendría complejidad constante. 

\begin{algorithm}[H]
$T_{1,1}$ $\leftarrow$ $M_{1,1}$\;
\For{$i$ desde 2 hasta $C$}{
	$T_{i,1}$ $\leftarrow$ $M_{i,1} + T_{i-1,1}$\;
}
\For{$j$ desde 2 hasta $A$}{
	$T_{1,j}$ $\leftarrow$ $M_{1,j} + T_{1,j-1}$\;
}
\For{$i$ desde 2 hasta $C$}{
\For{$j$ desde 2 hasta $A$}{
	$T_{i,j}$ $\leftarrow$ $M_{i,j} + T_{i,j-1} + T_{i-1,j} - T_{i-,j-1}$\;
}
}
\caption{Precómputo de la tabla aditiva}
\end{algorithm}

Podemos ver que cada valor de $T$ lo calculamos con una cantidad constante 
de operaciones aritméticas, por tanto computar cada valor tiene complejidad constante. 
Como $T$ es de $C \times A$ al igual que $M$, el precómputo tiene complejidad 
$\bigO(AC)$.


\subsubsection*{Resolución de las queries}
Como dijimos antes, la idea será usar la tabla $T$ con las sumas precomputadas 
para poder responder las queries en tiempo constante. Dada una query con valores 
$c_1, a_1, c_2, a_2$, la respuesta que buscamos es la suma de valores de $M$ en el 
rectángulo desde $(c_1+1, a_1+1)$ a $(c_2,a_2)$. 

En principio podemos tomar $T_{c_2,a_2}$, que contiene la sumatoria de los 
elementos de $M$ hasta $(c_2,a_2)$, pero debemos quitarle varias secciones. 

Si queremos restar los elementos de $M$ que están en las filas 1 a $c_1$ 
(entre la columnas 1 y $a_2$) y los que están en las columnas 1 a $a_1$ 
(entre las filas 1 y $c_2$), podemos restar entonces $T_{c_1,a_2}$ y $T_{c_2,a_1}$. 

\begin{equation}
T_{c_2,a_2} - T_{c_1,a_2} - T_{c_2,a_1} = \sum\limits_{i = 1}^{c_2} \sum\limits_{j = 1}^{a_2} M_{i,j} - \sum\limits_{i = 1}^{c_1} \sum\limits_{j = 1}^{a_2} M_{i,j} - \sum\limits_{i = 1}^{c_2} \sum\limits_{j = 1}^{a_1} M_{i,j}
\end{equation}

\begin{equation}
T_{c_2,a_2} - T_{c_1,a_2} - T_{c_2,a_1} = \sum\limits_{i = c_1+1}^{c_2} \sum\limits_{j = a_1+1}^{a_2} M_{i,j} - \sum\limits_{i = 1}^{c_1} \sum\limits_{j = 1}^{a_1} M_{i,j} 
\end{equation}

Nuevamente nos pasa algo similar a cuando hacíamos el precómputo de la tabla $T$, 
ahora tenemos una sección de elementos que estamos restando dos veces, que es el 
rectángulo que va hasta $(c_1,a_1)$. Pero sabemos que esta sección la tenemos calculada 
en $T_{c_1,a_1}$, así que podemos sumarla al resultado que teníamos antes, para evitar 
restarla dos veces. Entonces tenemos: 

\begin{equation}
T_{c_2,a_2} - T_{c_1,a_2} - T_{c_2,a_1} + T_{c_1,a_1} = \sum\limits_{i = c_1+1}^{c_2} \sum\limits_{j = a_1+1}^{a_2} M_{i,j}
\end{equation}

que es exactamente el resultado que buscábamos. 

El algoritmo para responder las queries sería el siquiente entonces 


\begin{algorithm}[H]
\For{cada query $(c_1, a_1, c_2, a_2)$}{
	\Return $T_{c_2,a_2} - T_{c_1,a_2} - T_{c_2,a_1} + T_{c_1,a_1}$\;
}
\caption{Proceso de las queries}
\end{algorithm}

Así responder cada query tiene complejidad constante porque solo hacemos 
una cantidad de operaciones aritméticas. En total, responder $Q$ queries tendrá 
complejidad $\bigO(Q)$. \\

En total, todo el algoritmo tiene la complejidad del precómputo más el costo de 
responder las queries, o sea $\bigO(AC + Q)$, como queríamos. 

\newpage
\subsection{Código de la solución}
\lstinputlisting[numbers = left]{../src/ej3/ej3.cpp}
