\newpage{}
\section{Problema D:\@ Diversión asegurada}
\textbf{Peso del ejercicio: 11 }

\newcommand{\mappend}{\operatorname{\bullet}}
\newcommand{\mempty}{\mathbf{\varepsilon}}
\newcommand{\sizeV}{\left|V\right|}

\subsection{Introducción Teórica}

\subsubsection{Monoides}

Un monoide es una estructura algebraica que contiene los siguientes elementos.

\begin{itemize}
	\item Un conjunto de valores \(\mathbb{S}\).
	\item Una operación asociativa \(\mappend : \mathbb{S} \times \mathbb{S} \rightarrow \mathbb{S}\).
	\item Un elemento \(\mempty : \mathbb{S}\) que sea nulo respecto a \(\mappend\).
\end{itemize}

En definitiva, la estructura debe cumplir las siguientes reglas.

\[
\begin{aligned}
	\forall a, b, c \in \mathbb{S}&. \quad a \mappend b \mappend c = (a \mappend b) \mappend c = a \mappend (b \mappend c) \\
	\forall a \in \mathbb{S}&. \quad a \mappend \mempty = \mempty \mappend a = a
\end{aligned}
\]

Generalizar este tipo de estructuras ayuda a simplificar el resto de la explicación del problema.

\subsubsection{Segment Tree}

Dado una secuencia de monoides \(V : \left[\left<\mathbb{S}; \mappend; \mempty\right>\right]\), un segment tree es un árbol binario \(A\) con un elemento \(A_v\) que cumple con las siguientes propiedades:

\begin{itemize}
	\item Para cada \(a\) hoja de a \(A\), \(a_v\) corresponde a un elemento \(V\).
	\item \(A\) debe estar \textbf{lleno}: todos los nodos deben tener exactamente \(0\) ó \(2\) hijos.
	\item Cualquier elemento no-hoja \(f \in A\) tiene hijos \(f_l, f_r \in A\), y su valor es igual a \(f_v = {f_l}_v \mappend {f_r}_v\).
\end{itemize}

Adicionalmente, para simplificar la estructura agregamos dos reglas.

\begin{itemize}
	\item \(A\) debe ser \textbf{perfecto}: todas las hojas deben estar al mismo nivel.
	\item \(A\) contiene \(\sizeV\) hojas con elementos correspondientes a \(V\), y \(r\) hojas con elementos correspondientes a \(\mempty\), donde \(|V| + r\) es una potencia de 2.\footnote{Notar que podríamos representar a cada elemento del árbol con valor opcional y de esta manera no necesitar una estructura con un elemento nulo, pero ¡los opcionales ya son monoides!}
\end{itemize}

\subsubsection{Range Minimum Query}

Dado un segment tree \(A\), cualquier elemento no-hoja \(f \in A\) que contiene hijos \(l, r \in A\) los cuales corresponden a elementos de una secuencia de monoides \(V\) o a \(\mempty\), va a tener el valor \(l \mappend r\). Esta propiedad se puede extender al elemento \(p\) padre de \(f\), que si contiene \(4\) nietos \(l, r, q, w \in A\), \(p = (l \mappend r) \mappend (q \mappend w)\). Como \(\mappend\) es una operación asociativa, \(p = l \mappend r \mappend q \mappend w\).

De hecho, si la altura de \(A\) es \(h\), cualquier elemento de altura \(j\) va a representar la reducción de \(\mappend\) en \(2^{h - j}\) elementos de \(V\) (o de \(\mempty\)).

Cualquier subsecuencia \(V_i, \dots, V_j\) puede ser representada combinando elementos de \(A\). Tomando \(l(n)\) y \(r(n)\) como los límites de los números de elementos combinados en el nodo \(n \in A\) y \(r\) como su raíz, se puede definir el reductor \(R : A \times \mathbb{N} \times \mathbb{N} \rightarrow S\), que representa la combinación de los elementos entre un par de valores que estén bajo un nodo de la siguiente forma.

\[
\begin{split}
	& V_i \mappend \cdots \mappend V_{j - 1} = R(r, i, j) \\
	& R(n, i, j) =
	\begin{cases}
		\mempty &\text{si } \left[l(n), r(n)\right) \cap \left[i, j\right) = \varnothing \\
		n_v &\text{si } \left[l(n), r(n)\right) \supseteq \left[i, j\right) \\
		R(n_l) \mappend R(n_r) &\text{sino}
	\end{cases}
\end{split}
\]

Se puede probar correctitud de una manera simple: si el rango buscado y el de \(n\) son disjuntos, entonces el resultado es vacío. Si todos los valores bajo \(n\) corresponden al rango buscado, no tiene sentido bajar a los nodos hijos.

Como todos los valores de \(V\) están representados en \(A\), y todos los nodos se pueden llegar pasando por los hijos de los elementos desde la raíz, la función \(R\) y el algoritmo que lo simule puede combinar los elementos entre cualquier par de números.

\subsection{Complejidad en tiempo}

El árbol \(A\) contiene a cada elemento de \(V\) como hoja. Como es un árbol binario, y a pesar de ser perfecto no tiene más niveles que el árbol mínimo que contiene a estos valores, su tamaño y el tiempo de construcción es \(\bigO(\sizeV)\).

Mirando ingenuamente la definición de \(R\) uno podría pensar que cada query tiene tiempo \(\bigO(2^{\sizeV})\). Sin embargo, es fácil ver que no gracias a algunas propiedades de la query.

Si se calcula \(R\) en dos nodos diferentes \(a, b \in A\) que tengan la misma altura en el árbol:
\begin{itemize}
	\item Los dos nodos no van a ser hermanos, ya que en ese caso se calcularía \(R\) para su padre.
	\item No puede existir un nodo \(c \in A\) al que se le calcule \(R\) que tenga la misma altura que \(a\) y \(b\) y que \(a < c < b\), ya se está calculando la reducción de un rango que se contiene los valores de \(a\) y de \(b\), por lo cual también se debería calcular \(R\) para el hermano de \(c\), y por la regla anterior esto es innecesario.
\end{itemize}

De esta manera se puede ver que \textbf{se calculan a lo sumo 2 valores por cada nivel del árbol}. Como \(A\) es un árbol binario con \(\bigO(\sizeV)\) nodos, tiene \(\bigO(\log \sizeV)\) niveles. Por lo tanto, calcular cada query tiene complejidad \(\bigO(\log \sizeV)\), y construir un árbol con \(\sizeV\) valores y hacer \(q\) queries tiene complejidad final \(\bigO(\sizeV + q \cdot \log \sizeV)\).

\subsection{Aplicación al problema}

En el problema, se puede definir el monoide \(\left[\left<\mathbb{S}; \mappend; \mempty\right>\right]\) como

\[
\begin{split}
	\mathbb{S} &= \left<\mathbb{N}, \mathbb{N}\right> \\
	\left<q, w\right> \mappend \left<a, s\right> &= \left<\operatorname{max}(q, w, a, s), \operatorname{2^{nd}max}(q, w, a, s)\right> \\
	\mempty &= \left<0, 0\right>
\end{split}
\]

De esta manera, este representa a la mayor y la segunda mayor diversión de un conjunto de días, y su elemento nulo es 0 ya que ningún día puede tener diversión negativa.

Se arma un Segment Tree con estos valores, y por cada query \(\left<P_1, U_1\right>, \dots, \left<P_R, U_R\right>\) se busca combinar los valores entre los elementos \(P_n\) y \(U_n\), que luego se suman. Por la fórmula anterior, esto tiene complejidad \(\bigO(D + R \cdot \log D)\).

\subsection{Casos de Prueba}

\centerline{%
\begin{tabulary}{1.1\textwidth}{|L|l|L|}
\hline
\textbf{Caso de prueba} & \textbf{Archivo de entrada} & \textbf{Salida esperada} \\
\hline
Casos de prueba chicos hechos a mano & \lstinputlisting[basicstyle=\footnotesize]{../src/ej4/ej4.in} & \lstinputlisting[basicstyle=\footnotesize]{../src/ej4/ej4.out} \\ \hline
La cantidad de días es una potencia de 2 & \lstinputlisting[basicstyle=\footnotesize]{../src/ej4/ej4_pot2.in} & \lstinputlisting[basicstyle=\footnotesize]{../src/ej4/ej4_pot2.out} \\ \hline
La cantidad de días es una potencia de 2 más 1 & \lstinputlisting[basicstyle=\footnotesize]{../src/ej4/ej4_pot2m1.in} & \lstinputlisting[basicstyle=\footnotesize]{../src/ej4/ej4_pot2m1.out} \\ \hline
Todos los queries son del menor tamaño posible & \lstinputlisting[basicstyle=\footnotesize]{../src/ej4/ej4_borde.in} & \lstinputlisting[basicstyle=\footnotesize]{../src/ej4/ej4_borde.out} \\ \hline
Muchos días del año y muchos queries & \textit{ej4\_grande.in} (\(R = 10^5, D = 10^5\)) & El programa termina en \(0.5\) segundos y sin errores. \\ \hline
\end{tabulary}
}

\subsection{Código de la solución}
\lstset{inputencoding=utf8/latin1}
\lstinputlisting[numbers = left]{../src/ej4/ej4.cpp}
