\newpage{}
\section{Problema D:\@ Diversión asegurada}
\textbf{Peso del ejercicio: }

\newcommand{\mempty}{\mathbf{\varepsilon}}

\subsection{Introducción Teórica}

\subsubsection{Monoides}

Un monoide es una estructura algebraica que contiene los siguientes elementos.

\begin{itemize}
	\item Un conjunto de valores \(\mathbb{S}\).
	\item Una operación asociativa \(\bullet : \mathbb{S} \times \mathbb{S} \rightarrow \mathbb{S}\).
	\item Un elemento \(\mempty : \mathbb{S}\) que sea nulo respecto a \(\bullet\).
\end{itemize}

En definitiva, la estructura debe cumplir las siguientes reglas.

\[
\begin{aligned}
	\forall a, b, c \in \mathbb{S}&. \quad a \bullet b \bullet c = (a \bullet b) \bullet c = a \bullet (b \bullet c) \\
	\forall a \in \mathbb{S}&. \quad a \bullet \mempty = \mempty \bullet a = a
\end{aligned}
\]

Generalizar este tipo de estructuras ayuda a simplificar el resto de la explicación del problema.
