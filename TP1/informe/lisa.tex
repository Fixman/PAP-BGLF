\newpage{}
\section{El cumpleaños de Lisa}

\subsection{Descripción del Problema}

El enunciado puede resumirse de la siguiente forma. Inicialmente tenemos un conjunto de $N$ amigas de Lisa (donde $ 1 \leq N \leq 18$) y además se tiene una matriz, llamémosla $D \in \mathbf{Z}^{N \times N}$, donde $D_{i,j}$ indica la diversión que aporta la pareja de amigas $A_i$ y $A_j$ en caso de concurrir a una misma fiesta (notar que puede ser negativa).

Un conjunto de amigas nos define una fiesta, y se define la diversión de ese conjunto (fiesta) como la suma de las diversiones de todas las parejas de amigas del conjunto sin importar el orden, o sea, que la diversión que aportan $A_i$ con $A_j$ y $A_j$ con $A_i$ se cuenta solo una vez.

Ahora, podemos definir la diversión de una partición, como la suma de las diversiones de los subconjuntos de la partición.

Teniendo todo esto definido, la consigna del problema se resume en hallar la diversión óptima que puede tener una partición del conjunto de $N$ amigas original. 

\subsection{Solución propuesta al problema}

La idea será describir la solución para un conjunto de amigas en función de sus subconjuntos. Para ello, notemos primero las siguientes observaciones:

Vamos a suponer que tenemos una función $g$ que resuelve una instancia al problema dado (o sea, toma un conjunto de amigas con sus respectivas diversiones y calcula el valor óptimo de diversión que puede tener una partición del conjunto original).

\begin{itemize}
	\item \textbf{Observación 1} : La solución para un conjunto con una sola amiga es $0$. 

	\textit{Demostración} : Dicha amiga no tiene otra amiga para aportar diversión (por enunciado, Lisa no aporta diversión a una fiesta).
	\item \textbf{Observación 2} : Dado un conjunto $\mathcal{A}$ de amigas,   la solución, $g(\mathcal{A})$, puede escribirse como $g(\mathcal{B}) + g(\mathcal{C})$, para algún par $\mathcal{B}$ y $\mathcal{C}$, con $\mathcal{B},\mathcal{C} \subset \mathcal{A}$ y donde $\mathcal{B} \overset{d}{\cup} C = \mathcal{A}$ (unión disjunta).
	
	\textit{Demostración} : Si el valor óptimo se obtiene haciendo una sola fiesta con todas las amigas de $\mathcal{A}$, podemos tomar $\mathcal{B} = \mathcal{A}$ y $C = \emptyset$. Por otro lado, si el valor óptimo se obtiene haciendo dos o más fiestas, podemos tomar una de esas fiestas como $\mathcal{B}$, y la unión de todas las amigas en fiestas restantes como $\mathcal{C}$. Luego, como amigas que asisten a fiestas distintas no aportan diversión alguna, $g(\mathcal{A}) = g(\mathcal{B}) + g(\mathcal{C})$.
\end{itemize}

En la Observación 2 se asume, aunque no se explicita, la existencia de una solución óptima al problema para un conjunto finito dado, porque hay solo finitas particiones de un conjunto. De todas formas, esto puede verse trivialmente, dado que alcanza con tomar $N := \# \mathcal{A}$ fiestas distintas para separar a todas las amigas, para luego acotar la cantidad de particiones burdamente por $N^N < \infty $.

La importancia de la segunda observación radica en que ahora podemos deducir la siguiente fórmula para $\boxed{g(\mathcal{A}) = \underset{\mathcal{B} \overset{d}{\cup} C = \mathcal{A}}{\text{max}} g(\mathcal{B}) + g(\mathcal{C})}$. Notemos entonces que para cada elección de $\mathcal{B}$, nos queda determinado $C = \mathcal{A} \setminus \mathcal{B}$, y solamente hay $2^N$ elecciones posibles de $\mathcal{B}$.

\begin{itemize}
	\item \textbf{Observación 3} : la elección $\mathcal{B} = \mathcal{B}_0$ y $\mathcal{C} = \mathcal{A} \setminus \mathcal{B}_0$ es idéntica a $\mathcal{B} = \mathcal{A} \setminus \mathcal{B}_0$ y $\mathcal{C} = \mathcal{B}_0$, pues basta renombrar $\mathcal{B}$ y $\mathcal{C}$. Esto nos da que basta en realidad con $2^{N-1}$ formas de elegir $\mathcal{B}$.
\end{itemize}

Esto nos da la pauta de la recursión clave que nos permitirá resolver el problema. Si uno lo ve con ojos más implementativos, debería hacernos ruido en la fórmula obtenida, que estaríamos llamando recursivamente a $g(\mathcal{A})$ para calcular $g(\mathcal{A})$. Ahora, solamente lo utilizamos (observación 2) si el óptimo se alcanza haciendo una fiesta con todas las amigas de $\mathcal{A}$. Si llamamos entonces $d$ a la función que calcula la diversión que aporta un conjunto de amigas, nos queda:

$$ \boxed{g(\mathcal{A}) =  \text{max} \left \{ d(\mathcal{A}),\underset{\mathcal{B} \subset \mathcal{A}, \mathcal{B} \neq \emptyset, \mathcal{C} = \mathcal{A} \setminus \mathcal{B}}{\text{max}} g(\mathcal{B}) + g(\mathcal{C}) \right \} } \quad  (\mathbf{A}) $$


\subsection{Detalles de implementación}

Primero que nada debemos decidir cómo representar a nuestros conjuntos en el problema. Llamemos al igual que antes $ N := \# \mathcal{A}$, luego, vamos a usar el desarrollo binario (ampliado con ceros hasta $N$ cifras) de los números desde $0$ hasta $2^N$ (inclusive el $0$, exclusive $2^N$) para representar a todo subconjunto de $\mathcal{A}$. Lo que vimos en clase con el nombre de \textbf{máscara de bits}.

\begin{itemize}
	\item \textbf{Ejemplo} : Si $\mathcal{A} = \left \{ A_1, A_2, A_3, A_4 \right \}$, entonces $ N = 4$, y si queremos representar al subconjunto $ \left \{ A_2, A_4 \right \}$, lo representamos con el número $5$, pues $(5)_2 = 0101$, donde la $i-$ésima posición de izquierda a derecha representa si está o no la amiga $A_i$, por ejemplo, la cifra menos significativa, representa que $A_4$ se encuentra en el subconjunto.
\end{itemize}

Originalmente tenemos al conjunto $\mathcal{A}$ con todas las amigas de Lisa, lo cual se representa con el número $2^N-1$. Al llamar a la recursión dada por $(\mathbf{A})$ sucesivamente, hay muchas instancias que se solapan, y no queremos tener que volver a calcularlas nuevamente, pues sería ineficiente. Esto nos da la pauta de que podemos utilizar \textbf{programación dinámica} para no recalcular $g(\mathcal{J})$ muchas veces (con $\mathcal{J}$ algún subconjunto de $\mathcal{A}$).

\begin{itemize}
	\item \textbf{Ejemplo} : Consideremos la siguiente instancia posible.
	
	 $$\overset{\mathcal{B}_1}{\overbrace{100}} \overset{\mathcal{C}_1}{\overbrace{\underset{\mathcal{J}}{\underbrace{111111}}0001}}$$
	
	$$\underset{\mathcal{B}_2}{\underbrace{100\overset{\mathcal{J}}{\overbrace{111111}}}}\underset{\mathcal{C}_2}{\underbrace{0001}}$$
	
	En este caso, para resolver $g(\mathcal{C}_1)$ tuvimos que calcular (entre otras cosas) $g(\mathcal{J})$. Si no nos guardamos este resultado, luego, cuando se quiera calcular $g(\mathcal{B}_2)$ tendríamos que calcular $g(\mathcal{J})$ nuevamente, evidenciando la ineficiencia.
\end{itemize} 

\subsection{Implementación del Algoritmo}

\subsubsection{Variables globales}

Por simplicidad, en las funciones a definir se consideran como globales (y no se pasan como parámetros) a:

\begin{enumerate}

	\item La matriz $D \in \mathbf{Z}^{N \times N}$ que indica la diversión que aporta cada pareja de amigas en una misma fiesta.
	\item Un arreglo de enteros de largo $2^N$ que llamaremos $\textsc{dp}$. En $\textsc{dp}(i)$ al finalizar el algoritmo quedará guardada la solución al problema del conjunto que representa la máscara de bits $i$.
\end{enumerate}  

\subsubsection{Funciones Auxiliares}

\begin{itemize}
	\item \texttt{cantidadAmigas} $ : \mathbf{Z} \rightarrow \mathbf{Z}$
	
	Dado un entero \textsc{mask}, que representa un conjunto de amigas calcula la cantidad de amigas que hay en el conjunto representado (o sea, la cantidad de bits encendidos).

	\item \texttt{diversion} $ : \mathbf{Z} \rightarrow \mathbf{Z}$
	
	Dado un entero \texttt{mask}, que representa un conjunto de amigas calcula la diversión del conjunto (como la suma de las diversiones que aporta cada pareja).
	
	\item \texttt{best} $ : \mathbf{Z} \rightarrow \mathbf{Z}$
	
	Calcula la solución a la instancia dada por el conjunto que representa \textsc{mask}, utilizando la recursión $(\mathbf{A})$ vista en la sección anterior, y a la vez llena al arreglo \textsc{dp} según su especificación.
\end{itemize}

\subsubsection{Algoritmo}

\begin{enumerate}
	\item Cargamos los datos de entrada (el entero $N$ y la matriz $D \in \mathbf{Z}^{N \times N}$, e inicializamos el arreglo \texttt{dp} con largo $2^N$ y un $-1$ en cada lugar.
	\item Llamamos a \texttt{best}$(2^N - 1)$ y devolvemos su respuesta. 
\end{enumerate}

\begin{itemize}
	\item \textbf{Comentario 1} : Notar que $(2^N-1)_2 = \overset{N}{\overbrace{111\dots111}}$, representa el conjunto con todas las amigas
	\item \textbf{Comentario 2} : Haciendo $N$ fiestas separadas (una por amiga), se consigue una diversión total de $0$, por lo que $-1$ jamás será solución a una instancia del problema.
\end{itemize}

 
 Como podemos ver, el algoritmo en sí radica en la implementación de la función $\texttt{best}$

\subsection{Implementación y Análisis de Complejidad}
\begin{itemize}
	\item \texttt{cantidadAmigas}$(\textsc{mask} \in \mathbf{Z})$
	
	$\textsc{ans} \leftarrow 0$
	
	$\texttt{mientras}$ $\textsc{mask} > 0:$

	$\begin{matrix}
	& \textsc{ans} & \leftarrow  & \textsc{ans} + \left [ \textsc{mask} \pmod{2}\right ] \\ 
	& \textsc{mask} & \leftarrow  & \left \lfloor \frac{\textsc{mask}}{2} \right \rfloor 
	\end{matrix}$
	
	$\texttt{devolver} \; \textsc{ans}$
\end{itemize}	
	Como en cada iteración \textsc{mask} se divide por dos, y se realizan cálculos en $\mathcal{O}(1)$, la función nos queda $\mathcal{O}(\text{log}_2(\textsc{mask}))$. Usando que $ \textsc{mask} \leq 2^N $, la complejidad nos queda $\mathcal{O}(N)$.
\begin{itemize}
	
	\item \texttt{diversion} $ (\textsc{mask} \in \mathbf{Z}) $
	
	$\textsc{indiceAmigas} \leftarrow [ \; ]$ \textit{(lista vacía)}
	
	$\textsc{indice} \leftarrow N$ 
	
	$\texttt{mientras} \; \textsc{mask} > 0:$
	
	\quad $ \texttt{si} \; \textsc{mask} \equiv 1 \pmod{2}$ 
	
	\quad \quad $ \textsc{indiceAmigas.agregar(indice)}$
	
	\quad $\textsc{indice} \leftarrow \textsc{indice} - 1$
	
	\quad $\textsc{mask} \leftarrow \left \lfloor \frac{\textsc{mask}}{2} \right \rfloor$ 
	
	$\textsc{ans} \leftarrow 0$
	
	\texttt{para cada par } (i,j) \texttt{ de amigas en} \textsc{indiceAmigas}:
	
	\quad $\textsc{ans} \leftarrow \textsc{ans} + D_{i,j}$
	
	\texttt{devolver} \textsc{ans}

\end{itemize}

	El análisis de la primer parte de la función es análogo al caso anterior. La complejidad de la segunda parte, es cuadrática en el largo de $\textsc{indiceAmigas}$, como $\# \textsc{indiceAmigas} \leq N$, en total la función nos queda con una complejidad de $\mathcal{O}(N) + \mathcal{O}(N^2)$, lo cual es $\mathcal{O}(N^2)$.
	
\begin{itemize}
	
	\item $\texttt{best}(\textsc{mask} \in \mathbf{Z})$
	
	$\texttt{si } \textsc{dp(mask)} = -1:$
	
	\quad 	$\texttt{si cantidadAmigas}(\textsc{mask}) \leq 1:$
	
	\quad \quad $ \textsc{dp(mask)} \leftarrow 0$
	
	\quad \texttt{sino}:
	
	\quad \quad $ \textsc{dp(mask)} \leftarrow \texttt{diversion}(\textsc{mask})$
	
	\quad \quad $ \texttt{para cada } \textsc{mask}_1 \texttt{ que representa un sobconjunto propio del de } \textsc{mask}:$
	
	\quad \quad \quad $\texttt{si } \textsc{mask}_1 > \textsc{mask}_1^c:$
	
	\quad \quad \quad \quad $\textsc{dp(mask)} \leftarrow \text{max} \left  \{ \textsc{dp(mask)}, \texttt{best}(\textsc{mask}_1) + \texttt{best}(\textsc{mask}_1^c)  \right \} $
	
	$\texttt{devolver} \textsc{dp(mask)}$
	
\end{itemize}	

	Donde $\textsc{mask}_1^c$ indica al complemento de $\textsc{mask}_1$ relativo a $\textsc{mask}$. Por ejemplo, si $\textsc{mask} = 101101$, y $\textsc{mask}_1 = 100100$, entonces $\textsc{mask}_1^c = 001001$ (y NO es $011011$, que sería el complemento del conjunto que representa). Si se quiere se puede pensar como el complemento del conjunto, intersecado con el conjunto que representa $\textsc{mask}$.
	
	Otra aclaración es que la línea $" \texttt{si } \textsc{mask}_1 > \textsc{mask}_1^c:"$, se condice con la observación 3 que vimos más arriba en el trabajo.
	
	Finalmente, el análisis de complejidad para esta función radica en la cantidad de veces que se llama a la función \texttt{best} (pues luego se hace una comparación en $\mathcal{O}(1)$ al tomar máximo). Para ello, notemos que hay dos casos bien marcados en el algoritmo. 
	\begin{enumerate}
		\item $\textsc{dp(mask)} = -1$
		\item $\textsc{dp(mask)} \neq -1$
	\end{enumerate}

	Notemos que una vez realizado un llamado del primer caso, se actualiza el valor de $\textsc{dp(mask)}$, por un número mayor o igual que $0$ por el comentario 2 que vimos más arriba, de donde se desprende que en el próximo llamado a $\texttt{best}(\textsc{mask})$ estaremos en el segundo caso.	
	
	Por cada una de las $2^N$ máscaras distintas, se hace un llamado a $\texttt{best}$ por cada subconjunto propio del conjunto que representa, pero solo la primera vez que se llama a cada máscara se incurre en una complejidad de $\mathcal{O}(N^2)$, en el resto de los casos se resuelven en $\mathcal{O}(1)$. Sabiendo que hay $\binom{N}{k}$ máscaras que representan conjuntos de $k$ amigas, y que $\# \mathcal{P}(\mathcal{A}) = 2^{\# \mathcal{A}}$, por lo visto en clase, sumado a la sugerencia del trabajo práctico, nos queda que la complejidad del algoritmo es:
	$$ \mathcal{O} \left ( 2^N \cdot N^2 + \sum_{k = 1}^N \binom{N}{k} \cdot 2^k \right ) = \mathcal{O}\left (2^N \cdot N^2 + 3^N \right ) = \boxed{ \mathcal{O}(3^N)} $$
	
\subsection{C\'odigo de la soluci\'on}
\lstinputlisting[numbers = left]{../src/lisa/lisa.cpp}
