\newpage{}
\section{Experimentos nucleares}
\subsection{Soluci\'on teorica del problema}
El programa toma una lista de numeros enteros y encuentra la m\'axima suma
de numeros contiguos que se puede obtener, es decir, dado un arreglo
de $N$ numeros enteros $A_1,\dots, A_N$ encuentra $A_i+\dots+A_j$ ,$j>i$ 
m\'axima. Lo hace de manera golosa pues miro las sumas parciales, truncando
a cero cuando me paso al negativo, es decir, si $S_j$ es la suma parcial 
hasta el elemento $j$ entonces $S_{j+1}$ es el m\'aximo entre $S_j+A_{j+1}$ 
y $0$, hago esto porque no me conviene acarrear valores negativos.


Luego el m\'aximo de las sumas parciales truncadas, a partir de ahora
lo llamo $maxGlobal$, es la suma contigua m\'axima, pues sino ,es decir, 
existe $A_i+\dots +A_j$ ,$j>i$, que es mayor estricto que $maxGlobal$. 
Puedo suponer sin p\'erdida de generalidad que $A_i+\dots +A_k\geq0$ 
$\forall i\leq k\leq j$, pues si para alg\'un $k$ no lo \'es entonces 
$A_{k+1}+\dots + A_j>A_i+\dots +A_j$ y repito el argumento con 
$A_{k+1}+ \dots +A_j$ hasta obtener una suma contigua que no se anule en 
cada paso. Sea $s$ el m\'inimo \'indice tal que $A_s+\dots+A_k\geq0$ 
$\forall k\leq j$, entonces
\begin{equation}
	A_i+\dots +A_j \leq A_s+\dots+A_j = S_{s+j} \leq maxGlobal
\end{equation}
lo cual es un absurdo pues hab\'ia supuesto que $A_i+\dots +A_j>maxGlobal$,
luego $maxGlobal$ es la suma contigua m\'axima.

En el c\'odigo hacemos esto sin guardarnos todos los $S_j$ porque no los 
necesitamos, solo nos quedamos con el m\'aximo valor de las sumas parciales
truncadas en cada momento, $maxGlobal$, y la suma contigua truncada maxima
que termina en el valor actual, $maxEnding$.

\subsubsection{Correctitud de la implementaci\'on}
Hacemos inducci\'on en $N$ para ver que la implementaci\'on es correcta, 
formalmente lo que afirmo es que dado un arreglo de $N$ elementos, al 
final de la ejecuci\'on del c\'odigo $maxGlobal$ y $maxEnd$ guardan la 
suma m\'axima contigua de todo el arreglo y la suma parcial truncada maxima
que termina en $A_N$ respectivamente.

	Si $N=1$ entonces  $maxEnd$ toma el m\'aximo entre $A_1$ y $0$, que si
	$A_1>=0$ entonces $maxGlobal$ toma este valor sino queda en $0$. 
	Efectivamente cumplen el enunciado.
	
	
	Supongo que vale para $N-1$, veamos que vale para $N$.
	Por hip\'otesis inductiva despu\'es de $N-1$ iteraciones $maxGlobal$
	guarda la suma m\'axima global de los primeros $N-1$ elementos y 
	$maxEnd$ la suma m\'axima que termina en $A_{N-1}$. En la pr\'oxima 
	iteraci\'on $maxEnd$ guarda el valor de la suma m\'axima que termina 
	en $A_N$, si esta es mayor que $maxGlobal$ entonces $maxGlobal$ toma 
	este valor sino se queda con el valor anterior. Finaliza la 
	ejecuci\'on terminando $maxGlobal$ con el m\'aximo global de la 
	secuencia y $maxEnding$ con la suma m\'as grande que termina en 
	$A_N$, como queriamos.


La complejidad del codigo es $O(N)$ pues hay un solo bucle for que se
repite $N$ veces.
\subsection{C\'odigo de la soluci\'on}
\lstinputlisting[numbers = left]{../src/frink/frink.cpp}
