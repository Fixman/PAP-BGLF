\newpage{}
\section{Problema D: Dif\'icil de convencer}
\textbf{Peso del ejercicio: 9 }

\newcommand{\mappend}{\operatorname{\bullet}}
\newcommand{\mempty}{\mathbf{\varepsilon}}
\newcommand{\sizeV}{\left|V\right|}
\newtheorem{prop}{Proposici\'on}
\newtheorem{lema}{Lema}

\subsection{Entendiendo el problema}
	El enunciado nos pide que dada una permutaci\'on $P$ de $N$ elementos, 
digamos de cuantas formas pueden ser los resultados de las 
competencias tal que al final la tabla de resultados quede igual. Como
todos los equipos compiten contra todos en un principio tenemos un grafo
completo y luego si el $i$-esimo equipo le gano a $j$-esimo le asignamos 
la direccion $i->j$ a la arista que une el nodo $i$ con el $j$. 


	Toda configuraci\'on es representable de esta
manera , esto se conoce como grafo torneo, y todo grafo torneo se corresponde
con un resultado de una competencia. Luego si queremos que al
permutar los elementos bajo $P$ obtengamos un grafo isomorfo lo que necesitamos
es que $i->j$ entonces $P(i)>P(j)$. Reciprocamente si un grafo torneo
cumple que bajo la permutacion $P$ se mantiene invariante entonces
ese resultado de la competencia cumple lo requerido por el enunciado. Asi 
que veamos cuantos grafos torneo cumplen esto.

\paragraph{Notacion y aclaraciones:} P es una permutacion y aveces tambien
representa al grafo torneo sobre el cual actua. Notaremos aveces T al grafo
torneo y N a la cantidad de elementos de P.


\subsection{Ciclos pares, no los recomiendo}
	\begin{prop}\label{ciclopar} 
		Sea $P$ un ciclo de longitud par, entonces afirmo que no hay grafo 
torneo que se mantenga invariante por $P$.
	\end{prop}
	\begin{proof}
		Supongamos que existe $T$ grafo torneo invariante por $P$. Sin perdida de
generalidad podemos suponer que $|T|=N$, pues sino consideramos el subgrafo
torneo de $T$ afectado por $P$. Mas aun supongamos que
$P=(2,3,4,\dots , N, 1)$, pues sino renombro los nodos.

	
		Si $N=2k>2$ entonces sin perdida de generalidad $(1,1+k) \in E$. Luego
como $T$ es invariante por $P$ se sigue que 
$(k+1,1) = (P^k(1),P^k(k+1)) \in E$, lo cual es absurdo. 
	\end{proof}
	
		
	Si $P=P_1P_2\dots P_k$ la descomposicion de una permutacion en ciclos
disjuntos, entonces si alguno de los $P_i$ es un ciclo par de [\ref{ciclopar}] 
se deduce que no hay ningun grafo torneo invariante por P. Pues si asi
fuese el subgrafo torneo afectado por $P_i$ seria un grafo torneo 
invariante por $P_i$ lo cual es absurdo por [\ref{ciclopar}].
	
\subsection{Que onda los ciclos impares?}
	En esta secci\'on analizamos que pasa con los ciclos impares de una
permutaci\'on.


	\begin{lema}\label{cicloimpar}
		Sea $P$ un ciclo impar de longitud $N=2k+1$ entonces hay $2^k$ 
grafos torneo invariantes por $P$.
	\end{lema}
	\begin{proof}
		Fijada la direcci\'on de una arista, por ejemplo, $(1,2)\in E$
se sigue que tras aplicar consecutivamente $P$ a $(1,2)$ ,como el grafo es invariante
por $P$, estas aristas heredan la orientaci\'on dada por $(1,2)$,
es decir, $(P^s(1),P^s(2))\in E$ $\forall s$. 


		Luego considero la relaci\'on de equivalencia en $E$ dada por 
$(i,j)\sim (i',j')$ $\Leftrightarrow$ $\exists$ $s$ tal que 
$(P^s(i),P^s(j))=(i',j')$. Luego los elementos libres vienen dados por 
$E/\sim$ que son exactamente $\binom{N}{2}/N = (N-1)/2 = (2k+1-1)/2=k$, pues cada
clase de equivalencia tiene $N$ elementos y el numero de aristas en un
grafo torneo de $N$ elementos es $\binom{N}{2}$. Cada una de estas aristas 
puedo ser orientada de 2 formas por lo que el numero de grafos torneos 
distintos es $2^k$
	\end{proof}
	
	\begin{lema}\label{aristaciclos}
	Sea $P=P_1P_2$ una permutaci\'on compuesta por dos ciclos de longitud
$N_1$ y $N_2$ respectivamente, sin perdida de generalidad los ciclos son 
impares [\ref{ciclopar}]. Entonces la cantidad de maneras de dirigir las 
aristas de $P_1$ a $P_2$ es 
	\begin{equation}2^{mcd(N_1,N_2)}\end{equation}
	\end{lema}
	\begin{proof}
		Sean $a \in P_1$, $a' \in P_2$ y sin perdida de generalidad 
$(a,a')\in E$. Como $T$ es invariante por $P$ se sigue que $(P^s(a),P^s(a')) \in E$
$\forall s$, es decir, $(P_1^s(a),P_2^s(a'))\in E$ $\forall s$. Entonces
me restringo al conjunto de aristas que conectan $P_1$ con $P_2$, lo llamo
$E_t=\{ (i,i')\in E / i\in P_1, i'\in P_2 \}$ y alli tomo la
relacion de equivalencia dada por $(i,i')\sim(j,j') \iff \exists$ $s$ tal que
$(i,i')=(P_1^s(j),P_2^s(j'))$, observar que $|E_t|=N_1N_2$. 


		La cantidad de elementos en una clase de equivalencia viene dada por el 
minimo com\'un multiplo entre $N_1$ y $N_2$. Pues si $s_0$ es el m\'inimo 
$s$ tal que $(i,i')=(P^s(i),P^s(i'))$ entonces $i=P^{s_0}(i)$ y $i'=P^{s_0}(i')$, 
como $P_1$ es un ciclo de longitud $N_1$ se sigue que $N_1|s_0$ y 
$P_2$ es un ciclo de longitud $N_2$ entonces $N_2|s_0$. Entonces $mcm(N_1,N_2)|s_0$
, luego $mcm(N_1,N_2)\leq s_0$ y como $i=P_1^{mcm(N_1,N_2)}(i)$ y $i'=P_2^{mcm(N_1,N_2)}(i')$
por la minimalidad de $s_0$ se tiene que $s_0=mcm(N_1,N_2)$.


		Finalmente la cantidad de aristas libres son los elementos de 
$E_t/\sim$ que es $\frac{N_1N_2}{mcm(N_1,N_2)}=mcd(N_1,N_2)$. Como cada
arista la puedo orientar de dos maneras distintas se sigue que la cantidad
total de grafot torneo distintos es $2^{mcd(N_1,N_2)}$.
	\end{proof}
	
	\begin{prop}
	Sea $P=P_1\dots P_r$ una permutaci\'on compuesta por ciclos,
sin perdida de generalidad los ciclos son impares [\ref{ciclopar}]. Si
la longitud del ciclo $P_i$ es $N_i=2k_i+1$ entonces la cantidad de 
grafos torneo invariantes por $P$ es:
	\begin{equation}
		2^{\sum_{i=1}^{r}k_i + \sum_{1\leq i<j\leq r}^{}mcd(N_i,N_j)}
	\end{equation}
	\end{prop}
	\begin{proof}
		Hacemos induccion en r:
		\begin{itemize}
			\item si $r=2$, $E=E_1\cup E_2\cup E_{1,2}$ union disjunta ,donde 
$E_k=\{(i,j)\in E/ i,j \in P_k \}$ para $k=1,2$ y 
$E_{1,2} =\{(i,i')\in E / i\in P_1 , i'\in P_2 \}$. Consideramos 
ademas las relaciones de equivalencia definidas en [\ref{cicloimpar}] para 
$E_1$ y $E_2$ y en [\ref{aristaciclos}] para $E_{1,2}$. Entonces por los
lemas recien citados se sigue que la cantidad de grafos torneo invariantes
es el producto, es decir,
				\begin{equation}
					2^{k_1+k_2+mcd(N_1,N_2)}
				\end{equation}  
			
			\item si $r>2$, tomo $P'=P_1\dots P_{r-1}$, entonces por hipotesis
inductiva tengo que la cantidad de grafos torneo invariantes por $P'$ son
				\begin{equation}
					2^{\sum_{i=1}^{r-1}k_i + \sum_{1\leq i<j\leq r-1}^{}mcd(N_i,N_j)}
				\end{equation}
luego agregando $P_r$, por si solo por [\ref{cicloimpar}] agrego $2^{k_r}$ 
opciones. Por otro lado tengo la interacci\'on entre $P_r$ con $P_i$ ,
lo cual agrega $2^{mcd(N_r,N_i)}$ por [\ref{aristaciclos}] $\forall i<r$.
Juntando ambos terminos tengo que la cantidad total es la dada por la proposici\'on.
		\end{itemize}
	\end{proof}
\subsection{Implementaci\'on, el despertar de la fuerza}
	La implementacion se separa en dos partes, la primera en la cual
identificamos los ciclos disjuntos y guardamos la cantidad de veces que
aparece cada longitud. Una segunda parte en la cual calculamos los
maximos divisores comunes entre todas las longitudes y hacemos el
calculo final.
	
	
	Como entrada le pasamos el arreglo dado por la permutacion.
	% \incmargin{1em}
	% \restylealgo{boxed}\linesnumbered
	\begin{algorithm}
	\SetKwInOut{Input}{input}
	\SetKwInOut{Output}{output}
	\caption{Identificacion de ciclos}
	\Input{Arreglo de enteros $P$}
	\Output{Map de enteros a enteros}
	\BlankLine

	$n\leftarrow$ longitud de P\;
	$map<entero\rightarrow entero> apariciones$\;
	
	\While{Quedan elementos sin incertar}{
		$set<enteros> ciclo\leftarrow$ algun elemento $e$ de $P$ que no se incerto todavia en ningun ciclo;\
		
		
		\While{si $P[e]$ no esta en el ciclo}{
			agregar $P[e]$ al ciclo\;
			$e\leftarrow$ P[e]\;
		}
		$apariciones[\text{longitud del ciclo}]++$\;
	}
	return apariciones\;
	\end{algorithm}
	% \decmargin{1em}
	
	\paragraph{Para chequear la correctitud} supongamos que $P=P_1\dots P_r$
su descomposicion en ciclos disjuntos y $N_1,\dots ,N_r$ sus correspondientes
longitudes. Luego si $e$ es un elemento de $P_i$ entonces en $ciclo$ se va a guardar
todos los elementos de $P_i$ hasta que se repita $e$, que justamente en ese 
momento sale del bucle while y suma uno a esa ocurrencia de longitud.

	\paragraph{La complejidad} es inmediata por la condicion del bucle
while principal, que termina cuando no hay mas elementos para incertar
en algun ciclo. Entonces como mucho se repite $N$ veces, si $N$ es la cantidad
de elementos de la permutacion $P$. Internamente en cada iteracion del bucle
hacemos operaciones con set y map que tardan $Log(N)$ entonces la complejidad total
es $O(NLog(N))$;

	
	Como entrada le pasamos el map(entero,entero) previamente calculado.
	% \incmargin{1em}
	% \restylealgo{boxed}\linesnumbered
	\begin{algorithm}
	\SetKwInOut{Input}{input}
	\SetKwInOut{Output}{output}
	\caption{Calculo de grafos invariantes}
	\Input{Map(entero,entero) $apariciones$}
	\Output{Entero}
	\BlankLine
		
		$total\Leftarrow$ 1\;

		\For{\textbf{each} $longitud_1$ y $longitud_2$ distintas en $apariciones$}{
			$total*=2^{apariciones[longitud_1]*apariciones[longitud_2]*mcd(longitud_1,longitud_2)}$\;
		}
	
		\For{\textbf{each} $longitud=2k+1$ en $apariciones$}{
			$total*=2^{apariciones[longitud]k}$\;
		}
				
		return total;
	\end{algorithm}
	% \decmargin{1em}
	
	\paragraph{La correctitud} es inmediata por las proposiciones anterios,
el primer bucle for agrega las opciones dadas por las aristas entre los
ciclos. El segundo for agrega las opciones dadas por los mismos ciclos.
	\paragraph{Para analizar la complejidad} hay que pensar cuantos
numeros distintos impares puede haber que sumen $N$, pues como los ciclos
son impares y la suma de todas las longitudes es la longitud de $P$ entoces
esto daria una cota para ambos bucles, ya que el primero es cuadratico
en este numero y el segundo es lineal. Para ello hagamos el siguiente 
razonamiento, $N=\sum_{i=1}^{r}2k_i+1$, como quiero maximizar $r$ entonces
supongo que los $K_i$ empiezan desde $1$, entonces\\
\begin{math}
	N=\sum_{i=1}^{r}2i+1=2\sum_{i=1}^{r}k_i+r=2\frac{r(r+1)}{2}+r=r^2+2r
\end{math}\\
	De alli se despeja $r=-1+\sqrt{1+N}$.
	
	
	Luego el primer bucle for se repite $O(N)$ veces y el segundo $O(sqrt(N))$
veces. Ademas en el primero tenemos una potenciacion usando potenciacion
logaritmica la resolvemos en $O(log(apariciones[longitud_1]*apariciones[longitud_2]*mcd(longitud_1,longitud_2)))\leq O(log(N))$ y el mcd
tarda tambien $O(log(longitud_1)+log(longitud_2))\leq O(log(N))$. Por lo que el primer for
finalmente tarda $O(NLog(N))$ y el segundo $O(sqrt(N)Log(N))$

	\paragraph{Finalmente la complejidad} del problema esta acotada
por $O(NLog(N))$ como pedia el enunciado.

\subsection{Código de la solución}
\lstset{inputencoding=utf8/latin1}
\lstinputlisting[numbers = left]{../src/ej4/ej4.cpp}
